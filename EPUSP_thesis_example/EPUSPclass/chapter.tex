% -----------------------------------------------------------------------------
% chapter.tex
% 
% chapter environment definitions and formatting
% -----------------------------------------------------------------------------

% old comments for further revision
% ----------------------------------

%    Chapter and section: titles, page numbering, etc
%        This is a very delicated part. Let's give a flavour of what
%        happened. See comments in implementation for details.
%      redefining \chapter
%        - tests for determine begin of textual part and make changes at this
%          event.
%        - \pretextualchapter: a started chapter in the pre-textual part
%        - ProximoForaDoSumario -> switches off \ABNTaddcontentsline.
%        - Marks: care is they are inserted by default or not.
%        - redefinig \@schapter: now \chapter* is almost equal to \chapter,
%          appart from the title display. It also includes an entry in TOC.
%        - \@part and \@spart redefined: font purposes.
%        - defined \ABNTstartsection: stared section with centralized titles
%           and inclusion in TOC.
%        - new \@dottedtoline for compleance with NBR 6027-1989
%        - \@makechapterhead and \@makeschapterheadredefined to fit
%          norms. Paying attention to capchap and fonts.
%        - redefined all \section like commands (apply last redefinitions
%          and font purposes)
%        - \anexo and \apendice -> apply old \appendix changes and
%          replaces \@chapter by \@anapchapter. 
% Redefining \chapter and \part. The changes are
%  - \chapteritlepagestyle instead of explict plain page style
%  - \ABNTchaptermark instead of \chaptermark (must be before
%       \addcontentsline !!!) 
% - several commands were replaced by abntversions due to nuances of the
%   norms :-( it can create several uncompatibilities with packages that
%   change explictily these important commands, although I have tried very
%   hard to make as less damage on compatibility as possible. As a result
%   of this, package hyperref is totally broken with abnt.cls.
%
% What is going on here?
%   
%  This class supports auto include in toc for both \chapter* and
%  \section* (and similars). As we know, the command \chapter admits an
%  optional argument, which goes to the toc and the marks, if it's present
%  (otherwise the main parameter of \chapter is used in toc and marks also.)
%  
%  To allow exactly the same mechanism for \chapter and \chapter*, \section
%  and \section*, \subsection and \subsection*, etc, I had to define
%  similar commands to some important core commands in LaTeX, like \secdef
%  (we make a new one to avoid conflicts with other packages) and
%  \@startsection (I make a new one too).
%
%  Now, \ABNTstartsection, appart from make the stared version of a
%  section-like command to include the title in toc (if a boolean is set to
%  true), the title is centered for stared form.
%
%  Now, \chapter* and \part* are almost equal to \chapter and \part
%
%  The origin of this material are the files 
%     $texmfDIR/tex/latex/base/latex.ltx  (codigo fonte do LaTeX)
%     $texmfDIR/tex/latex/base/report.cls (classe standard)

% From the definition of \secdef -> stared version now admits optional
% param (after bug with hyperref, it was no longer used.)
\def\ABNTsecdef#1#2{\@ifstar{\@dblarg{#2}}{\@dblarg{#1}}}

% Changed explicit \thispagestyle{empty}
\renewcommand\chapter%
    {%
     \if@openright\cleardoublepage\else\clearpage\fi%
     \thispagestyle{\chaptertitlepagestyle}
     \ifthenelse{\boolean{ABNTinpretext}}%
       {%
        \ifthenelse{\boolean{ABNTaftertoc}}%
          {% change to textual part
           \setboolean{ABNTinpretext}{false}%
           \ABNTBeginOfTextualPart%
          }%
          {}%
       }% 
       {}%
     \global\@topnum\z@%
     \@afterindentfalse%
%     \ABNTsecdef\@chapter\@schapter%
     \secdef\@chapter\@schapter%
    }%

% To make tre-textual chapters (that fits page numbering squeme...)
% this is equal to \chapter*{}, but ignores in witch part is it. It does
% not change any tags.
\newcommand\pretextualchapter%
    {%
     \if@openright\cleardoublepage\else\clearpage\fi%
     \thispagestyle{\chaptertitlepagestyle}
     \global\@topnum\z@%
     \@afterindentfalse%
     \@schapter%
    }%

% Created useful \resetsubcounters from the code of \stepcounter.
\newcommand{\resetsubcounters}[1]{%
  \begingroup
    \let\@elt\@stpelt
    \csname cl@#1\endcsname
  \endgroup}

%% \ProximoForaDoSumario -> taking out some chapter or section from toc.

\newboolean{ABNTNextOutOfTOC}
\setboolean{ABNTNextOutOfTOC}{false}

%% \ProximoForaDoSumario[mark text]
% Now \ProximoForaDoSumario (re)sets the marks too. If one still want the
% marks, he/she must use it as the optional parameter of this command.

\newcommand{\ABNTnextmark}{}
\newcommand{\ProximoForaDoSumario}[1][]
  {
   \setboolean{ABNTNextOutOfTOC}{true}
   \renewcommand{\ABNTnextmark}{#1}
  }


\let\ABNToldaddcontentsline\addcontentsline\relax
\newcommand{\ABNTaddcontentsline}[3]%
  {\ifthenelse{\boolean{ABNTNextOutOfTOC}}
     {\setboolean{ABNTNextOutOfTOC}{false}}
     {\ifthenelse{\boolean{ABNThypertoc}}%
     {\addtocontents{#1}{\protect\contentsline{#2}{#3}{\thepage}{#2.\csname the#2\endcsname}}}%
     {\ABNToldaddcontentsline{#1}{#2}{#3}}%
     }}

\newcommand{\ABNTchaptermark}[1]
  {%
   \ifthenelse{\boolean{ABNTNextOutOfTOC}}
     {\markboth{\ABNTnextmark}{\ABNTnextmark}}
     {\chaptermark{#1}}%
  }

\newcommand{\ABNTsectionmark}[1]
  {%
   \ifthenelse{\boolean{ABNTNextOutOfTOC}}
     {\markright{\ABNTnextmark}}
     {\sectionmark{#1}}%
  }




% \@chapter : 
\def\@chapter[#1]#2%
      {\ABNTchaptermark{#1}% This command MUST came before addcontentsline
       \ifnum \c@secnumdepth >\m@ne
         \refstepcounter{chapter}%
         \typeout{\@chapapp\space\thechapter.}%
         \ifthenelse{\boolean{ABNTaftertoc}}
           {\ABNTaddcontentsline{toc}{chapter}%
                   {\protect\numberline{\thechapter}#1}}
           {}
       \else
         \ifthenelse{\boolean{ABNTaftertoc}}%
           {\ABNTaddcontentsline{toc}{chapter}{#1}}%
           {}%
       \fi       
       \if@twocolumn
         \@topnewpage[\@makechapterhead{#2}]%
       \else
         \@makechapterhead{#2}%
         \@afterheading
       \fi\par}

%% From the \@chapter code. Now \@schapter does almost the same as
%% \@chapter. Added \resetsubcounters.
\def\@schapter#1{%
      \if@twocolumn
        \@topnewpage[\@makeschapterhead{#1}]
      \else
        \@makeschapterhead{#1}
        \@afterheading
      \fi
      \@mkboth{#1}{#1}  % <-- inserted (must be before addcontentsline)
      \ifthenelse{\boolean{ABNTincludeintoc}}%
        {%
         \ifthenelse{\boolean{ABNTaftertoc}}
           {\ABNTaddcontentsline{toc}{chapter}{#1}}
           {}
        }%
        {}
      \resetsubcounters{chapter}\par
    }%

% Fonts in \@part and \@part* changed. Some \resetsubcounters added.
\def\@part[#1]#2{%
    \ifnum \c@secnumdepth >-2\relax
      \refstepcounter{part}%
      \ABNTaddcontentsline{toc}{part}{\protect\numberline{\thepart}#1}%
    \else
      \resetsubcounters{part}
      \ABNTaddcontentsline{toc}{part}{#1}%
    \fi
    \markboth{}{}%
    {\centering
     \interlinepenalty \@M
     \normalfont
     \ifnum \c@secnumdepth >-2\relax
       \ABNTchapterfont\ABNTchaptersize\ifthenelse{\boolean{ABNTcapchap}}%
          {\MakeUppercase{\partname~\thepart}}%
          {\partname~\thepart} %\partname~\thepart% Paulo Barreto
       \par
       \vskip 20\p@
     \fi
     \ABNTchapterfont\ABNTcapchapsize\ifthenelse{\boolean{ABNTcapchap}}%
          {\MakeUppercase{#2}}%
          {#2}\par}%
    \@endpart}

\def\@spart#1{%
    \ABNTaddcontentsline{toc}{part}{#1}%    
    \markboth{}{}%
    {\centering
     \interlinepenalty \@M
     \normalfont
     \ABNTchapterfont\ABNTcapchapsize\ifthenelse{\boolean{ABNTcapchap}}%
          {\MakeUppercase{#1}}%
          {#1}\par}%
    \@endpart}


% From \@startsection. The only difference is that it calls \@ssect
% changing the meaning id the first parameter. Now, instead of indentation,
% it gives section level for TOC purposes.
\def\ABNTstartsection#1#2#3#4#5#6{%
  \if@noskipsec \leavevmode \fi
  \par
  \@tempskipa #4\relax
  \@afterindenttrue
  \ifdim \@tempskipa <\z@
    \@tempskipa -\@tempskipa \@afterindentfalse
  \fi
  \if@nobreak
    \everypar{}%
  \else
    \addpenalty\@secpenalty\addvspace\@tempskipa
  \fi
  \@ifstar
    {\ABNTssect{#1}{#4}{#5}{#6}}% #3 replaced by #1 here
    {\@dblarg{\ABNTsect{#1}{#2}{#3}{#4}{#5}{#6}}}}

% I change the meaning of the first paramenter here. Instead of an indent
% skip, it is now the name of the section, for `toc' purposes.
\def\ABNTssect#1#2#3#4#5{%
  \@tempskipa #3\relax
  \ifdim \@tempskipa>\z@
    \begingroup
      #4{%
         \interlinepenalty \@M \centering
         \ifthenelse{\boolean{ABNTcapsec}}
             {\MakeUppercase{#5}}{#5}\@@par}%
    \endgroup
    \@ifundefined{ABNT#1mark}{}{\csname ABNT#1mark\endcsname{#5}}
    \ifthenelse{\boolean{ABNTincludeintoc}}
      {\ABNTaddcontentsline{toc}{#1}{#5}}
     {}
  \else
    \def\@svsechd{#4{#5}%
      \@ifundefined{ABNT#1mark}{}{\csname ABNT#1mark\endcsname{#5}}
      \ifthenelse{\boolean{ABNTincludeintoc}}%
         {\ABNTaddcontentsline{toc}{#1}{#5}}{}
    }%
  \fi
  \@xsect{#3}}

\def\ABNTsect#1#2#3#4#5#6[#7]#8{%
  \ifnum #2>\c@secnumdepth
    \let\@svsec\@empty
  \else
    \refstepcounter{#1}%
    \protected@edef\@svsec{\@seccntformat{#1}\relax}%
  \fi
  \@tempskipa #5\relax
  \ifdim \@tempskipa>\z@
    \begingroup
      #6{%
        \@hangfrom{\hskip #3\relax\@svsec}%
          \interlinepenalty \@M
          \ifthenelse{\boolean{ABNTcapsec}}
             {\MakeUppercase{#8}}{#8}\@@par}%
    \endgroup
    \@ifundefined{ABNT#1mark}{}{\csname ABNT#1mark\endcsname{#7}}
    \ABNTaddcontentsline{toc}{#1}{%
      \ifnum #2>\c@secnumdepth \else
        \protect\numberline{\csname the#1\endcsname}%
      \fi
      #7}%
  \else
    \def\@svsechd{%
      #6{\hskip #3\relax
      \@svsec \ifthenelse{\boolean{ABNTcapsec}}
             {\MakeUppercase{#8}}{#8}}%
      \@ifundefined{ABNT#1mark}{}{\csname ABNT#1mark\endcsname{#7}}
      \ABNTaddcontentsline{toc}{#1}{%
        \ifnum #2>\c@secnumdepth \else
          \protect\numberline{\csname the#1\endcsname}%
        \fi
        #7}}%
  \fi
  \@xsect{#5}}


% adding style for page numbers required by NBR 6027

\let\old@dottedtocline\@dottedtocline
\renewcommand{\@dottedtocline}[5]{%
  \ifthenelse{\boolean{ABNTpagenumstyle}}
     {%
      {\renewcommand{\@pnumwidth}{2.5em}%
       \renewcommand{\@tocrmarg}{3.5em}
       \old@dottedtocline{#1}{#2}{#3}{#4}%
             {\ifthenelse{\equal{#5}{}}{}{p.\thinspace#5}}}%
     }%
     {\old@dottedtocline{#1}{#2}{#3}{#4}{#5}}%
}



%% Font which chapter titles will be printed
\ifthenelse{\boolean{ABNTcapchap}}
  {\newcommand{\ABNTchapterfont}{\bfseries\upshape}}
  {\newcommand{\ABNTchapterfont}{\bfseries\itshape}}

\newcommand{\ABNTtocchapterfont}{\ABNTchapterfont\upshape}

%% Font which section titles will be printed
\newcommand{\ABNTsectionfont}{\bfseries}

%% In order to \MakeUppercase do not apply to math mode in chapter or
%% section titles, package textcase used
%
% Sorry, a bug on textcase adds extra space before text. User can fix it by
% him/herself.
%
%\ifthenelse{\boolean{ABNTcapchap}\or\boolean{ABNTcapsec}}
%  {\IfFileExists{textcase.sty}
%     {\RequirePackage[overload]{textcase}}
%     {}
%  }
%  {}

\newcommand{\ABNTchaptersize}{\normalsize} %{\Large}
\newcommand{\ABNTcapchapsize}{\normalsize} %{\Large}
\newcommand{\ABNTtitulosize}{\Large} %{\Large}

% defining how is typeset the \chapter
\def\@makechapterhead#1{%
  {%
%  \noindent\rule{\textwidth}{1.7pt}\\\par
  \normalfont\ABNTchaptersize\ABNTchapterfont%
  \espaco{simples}%
  \vspace*{\distHeaderChapter}
  \noindent%
  \parbox[b]{\textwidth}{%
    \parbox[t]{4ex}{\thechapter}% espaço entre o número e o título do capitulo
    \parbox[t]{\textwidth-4ex-1ex}%
      {\interlinepenalty\@M\raggedright%
        \ifthenelse{\boolean{ABNTcapchap}}%
          {\MakeUppercase{#1}}%
          {#1}
      }%
    \vspace*{\distAfterChapter}
    }\\[0pt]% era 2
  \vspace{\offsetAfterChapter}
  }%
}

\newif\ifotherhead
\otherheadfalse

% defining how is typeset the \chapter*
\def\@makeschapterhead#1{%
\ifotherhead
  % MUDA O COMPRIMENTO AQUI
  \vspace*{-1.2cm}\par
  {\centering\normalfont\ABNTchaptersize\ABNTchapterfont%
   \ifthenelse{\boolean{ABNTcapchap}}%
     {\MakeUppercase{#1}}%
     {#1} 
    \par}%
  \vspace{45pt}%
  \par
  \otherheadfalse % atribui falso para o boolean otherhead
\else
  \vspace*{0pt}\par
  {\centering\normalfont\ABNTchaptersize\ABNTchapterfont%
   \ifthenelse{\boolean{ABNTcapchap}}%
     {\MakeUppercase{#1}}%
     {#1}
    \par}%
  \vspace{45pt}%
  \par
\fi
}

% redefining to apply tocnumpageabnt
\renewcommand\l@chapter[2]{%
  \ifnum \c@tocdepth >\m@ne
    \addpenalty{-\@highpenalty}%
    \vskip 1.0em \@plus\p@
    \setlength\@tempdima{1.5em}%
    \begingroup
      \ifthenelse{\boolean{ABNTpagenumstyle}}
        {\renewcommand{\@pnumwidth}{3.5em}}
        {}
      \parindent \z@ \rightskip \@pnumwidth
      \parfillskip -\@pnumwidth
      \leavevmode \normalsize\ABNTtocchapterfont
      \advance\leftskip\@tempdima
      \hskip -\leftskip
      #1\nobreak\hfil \nobreak%
      \ifthenelse{\boolean{ABNTpagenumstyle}}
         {%
          \hb@xt@\@pnumwidth{\hss 
            \ifthenelse{\not\equal{#2}{}}{{\normalfont p.\thinspace#2}}{}}\par
         }
         {%
          \hb@xt@\@pnumwidth{\hss #2}\par
         }
      \penalty\@highpenalty
    \endgroup
  \fi}

% redefine to apply style
\renewcommand\part{%
  \if@openright\cleardoublepage\else\clearpage\fi%
  \thispagestyle{\chaptertitlepagestyle}%
  \if@twocolumn\onecolumn\@tempswatrue\else\@tempswafalse\fi%
  \null\vfil\secdef\@part\@spart}%

\renewcommand\l@part[2]{%
  \ifnum \c@tocdepth >-2\relax
    \addpenalty{-\@highpenalty}%
    \addvspace{2.25em \@plus\p@}%
    \begingroup
      \parindent \z@ \rightskip \@pnumwidth
      \parfillskip -\@pnumwidth
      \noindent{\leavevmode
       \ABNTtocchapterfont\large\noindent%
          #1\hfil \hb@xt@\@pnumwidth{\hss #2}}\par
       \nobreak
         \global\@nobreaktrue
         \everypar{\global\@nobreakfalse\everypar{}}%
    \endgroup
  \fi}

% formatting spaces for section, subsection, etc
% ----------------------------------------------

\renewcommand\section{\ABNTstartsection{section}{1}{\z@}%
                           {-3.5ex \@plus -1ex \@minus -.2ex}%
                           {2.3ex \@plus.2ex}%
                           {\espaco{simples}\normalfont%
                            \ABNTsectionfont\normalsize}}%large
\renewcommand\subsection{\ABNTstartsection{subsection}{2}{\z@}%
                           {-3.25ex\@plus -1ex \@minus -.2ex}%
                           {1.5ex \@plus .2ex}%
                           {\espaco{simples}\normalfont%
                            \ABNTsectionfont\normalsize}} %\large
\renewcommand\subsubsection{\ABNTstartsection{subsubsection}{3}{\z@}%
                           {-3.25ex\@plus -1ex \@minus -.2ex}%
                           {1.5ex \@plus .2ex}%
                           {\espaco{simples}\normalfont%
                            \ABNTsectionfont\normalsize}}
\renewcommand\paragraph{\ABNTstartsection{paragraph}{4}{\z@}%
                           {3.25ex \@plus1ex \@minus.2ex}%
                           {-1em}%
                           {\espaco{simples}\normalfont%
                            \ABNTsectionfont\normalsize}}
\renewcommand\subparagraph{\ABNTstartsection{subparagraph}{5}{\parindent}%
                           {3.25ex \@plus1ex \@minus .2ex}%
                           {-1em}%
                           {\espaco{simples}\normalfont%
                            \ABNTsectionfont\normalsize}}