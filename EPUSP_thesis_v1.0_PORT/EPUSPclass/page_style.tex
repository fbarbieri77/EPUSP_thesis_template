% -----------------------------------------------------------------------------
% page_style.tex
%
% definitions of
%	- fontstyle for header elements
%	- pagestyle "header"
%	- pagestyle plainheader
%	- pagestyle ruledheader
%	- indenting first paragraph of each section
%	- set the chaptertitlepagestyle
% -----------------------------------------------------------------------------


% fontstyle for header elements
% -----------------------------

\newcommand{\rightmarkformat}{\small\itshape}
\newcommand{\leftmarkformat}{\itshape}
\newcommand{\thepageformat}{\small}

\newcommand{\ABNTmarkboth}[2]{%
 \ifthenelse{\boolean{ABNTNextOutOfTOC}}
     {\markboth{\ABNTnextmark}{\ABNTnextmark}}
     {\markboth{#1}{#2}}%
}

\newcommand{\ABNTmarkright}[1]{%
 \ifthenelse{\boolean{ABNTNextOutOfTOC}}
     {\markright{\ABNTnextmark}}
     {\markright{#1}}%
}


% defining pagestyle "header"
% ---------------------------

\newcommand{\ps@header}{%
  \renewcommand{\@oddfoot}{}%
  \renewcommand{\@evenfoot}{}%
  \renewcommand{\@oddhead}%
    {{\rightmarkformat\rightmark}\hfill{\thepageformat\thepage}}%
  \renewcommand{\@evenhead}%
       {{\thepageformat\thepage}\hfill{\leftmarkformat\leftmark}}%
% Para \chapter* mostrar o cabecalho
  \let\@mkboth\ABNTmarkboth%
% Definindo a maneira como o comando o \chapter marca o cabecalho
  \renewcommand{\chaptermark}[1]{%
    \markboth%
       {\ifnum \c@secnumdepth >\m@ne%
            \thechapter{}  %
        \fi%
        ##1}%
       {\ifnum \c@secnumdepth >\m@ne%
            \thechapter{}  %
        \fi%
        ##1}%
  }%
  \renewcommand{\sectionmark}[1]{%
    \markright{%
      \ifnum \c@secnumdepth >\z@%
        \thesection\ \ %
      \fi%
      ##1}%
  }%   
}% 


% defining pagestyle plainheader
% ------------------------------

\newcommand{\ps@plainheader}{%
  \renewcommand{\@oddfoot}{}%
  \renewcommand{\@evenfoot}{}%
  \renewcommand{\@oddhead}{\hfill{\thepageformat\thepage}}%
  \renewcommand{\@evenhead}{{\thepageformat\thepage}\hfill}%
% Para \chapter* mostrar o cabecalho
  \let\@mkboth\ABNTmarkboth%
% Definindo a maneira como o comando o \chapter marca o cabecalho
  \renewcommand{\chaptermark}[1]{%
    \markboth%
       {\ifnum \c@secnumdepth >\m@ne%
            \thechapter\ \ %
        \fi%
        ##1}%
       {\ifnum \c@secnumdepth >\m@ne%
            \thechapter\ \ %
        \fi%
        ##1}%
  }%
  \renewcommand{\sectionmark}[1]{%
    \markright{%
      \ifnum \c@secnumdepth >\z@%
        \thesection\ \ %
      \fi%
      ##1}%
  }%   
}% 


% defining pagestyle ruledheader
% ------------------------------

\newcommand{\ps@ruledheader}{%
  \renewcommand{\@oddfoot}{}%
  \renewcommand{\@evenfoot}{}%
  \renewcommand{\@oddhead}%
     {\underline{\makebox[\textwidth]{\raisebox{-.5ex}{}%
       {\rightmarkformat\rightmark}\hfill{\thepageformat\thepage}}}}%
  \renewcommand{\@evenhead}%
     {\underline{\makebox[\textwidth]{\raisebox{-.5ex}{}%
       {\thepageformat\thepage}\hfill{\leftmarkformat\leftmark}}}}%
% Para \chapter* mostrar o cabecalho
  \let\@mkboth\ABNTmarkboth%
% Definindo a maneira como o comando o \chapter marca o cabecalho
  \renewcommand{\chaptermark}[1]{%
    \markboth%
       {\ifnum \c@secnumdepth >\m@ne%
            \thechapter\ \ %
        \fi%
        ##1}%
       {\ifnum \c@secnumdepth >\m@ne%
            \thechapter\ \ %
        \fi%
        ##1}%
  }%
  \renewcommand{\sectionmark}[1]{%
    \markright{%
      \ifnum \c@secnumdepth >\z@%
        \thesection\ \ %
      \fi%
      ##1}%
  }%   
}% 

% indenting first paragraph of each section
% -----------------------------------------

\ifthenelse{\boolean{ABNTindentfirst}}%
 {\RequirePackage{indentfirst}}%
 {}

% paragraph indentation size and skip
\setlength{\parindent}{\indentationSize}
\setlength{\parskip}{\indentationSkip}

\newcommand{\chaptertitlepagestyle}{plain}